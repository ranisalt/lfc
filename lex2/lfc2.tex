\documentclass[answers]{exam}
% rubber: module xelatex
% rubber: shell_escape
\usepackage{amsmath,amssymb}
\usepackage[a4paper,margin=2cm]{geometry}
\usepackage[pdf,tmpdir]{graphviz}
\usepackage{listings}
\usepackage{polyglossia}

\lstset{
    basicstyle=\itshape,
    literate={->}{$\rightarrow$}{2}
             {|}{$\mid$}{1}
}
\renewcommand{\solutiontitle}{}

\begin{document}
\setdefaultlanguage{brazilian}
\title{Linguagens Formais e Compiladores -- Lista 2}
\author{Ranieri S. Althoff -- 13100773}

\maketitle

\begin{questions}

\question Construa um autômato finito M tal que:

\begin{parts}

\part $T(M) = \{x \mid x \in a^n(b, c)^* \land n \geq 0 \land |x| \text{ seja
    ímpar} \land x \text{ não possui } bb \land x \text{ não possui } cc\}$

\begin{solution}
    \digraph[scale=0.5]{fsm1a}{
        rankdir=LR;

        node [shape = point]; qi;
        node [shape = doublecircle]; q1; q2; q3;

        node [shape = circle];
        qi -> q0;
        q0 -> q1 [label="a"];
        q1 -> q0 [label="a"];
        q0 -> q2 [label="b"];
        q0 -> q3 [label="c"];
        q1 -> q4 [label="b"];
        q3 -> q4 [label="b"];
        q4 -> q3 [label="c"];
        q1 -> q5 [label="c"];
        q2 -> q5 [label="c"];
        q5 -> q2 [label="b"];
    }
\end{solution}

\part $T(M) = \{x \mid x \in (0, 1)^+ \land x \text{ seja um binário divisível
    por } 6\}$

\begin{solution}
    \digraph[scale=0.5]{fsm1b}{
        rankdir=LR;

        node [shape = point]; qi;
        node [shape = doublecircle]; q0;

        node [shape = circle];
        qi -> q0;
        q0 -> q0 [label="0"];
        q0 -> q1 [label="1"];
        q1 -> q2 [label="0"];
        q1 -> q3 [label="1"];
        q2 -> q4 [label="0"];
        q2 -> q5 [label="1"];
        q3 -> q0 [label="0"];
        q3 -> q1 [label="1"];
        q4 -> q2 [label="0"];
        q4 -> q3 [label="1"];
        q5 -> q4 [label="0"];
        q5 -> q5 [label="1"];
    }
\end{solution}

\part $T(M) = \{x \mid x \in (1, 2, 3)^* \land \text{o somatório dos
    elementos de } x \text{ seja múltiplo de } 4\}$

\begin{solution}
    \digraph[scale=0.5]{fsm1c}{
        rankdir=LR;

        node [shape = point]; qi;
        node [shape = doublecircle]; q0;

        node [shape = circle];
        qi -> q0;
        q0 -> q1 [label="1"];
        q0 -> q2 [label="2"];
        q0 -> q3 [label="3"];
        q1 -> q2 [label="1"];
        q1 -> q3 [label="2"];
        q1 -> q0 [label="3"];
        q2 -> q3 [label="1"];
        q2 -> q0 [label="2"];
        q2 -> q1 [label="3"];
        q3 -> q0 [label="1"];
        q3 -> q1 [label="2"];
        q3 -> q2 [label="3"];
    }
\end{solution}

\part $T(M) = \{a^nyc^kxa^m \mid n,m,k \geq 1 \land x,y \in (a,b)^* \land \#a’s
    \text{ em } y \geq n \land \#a’s \text{ em } x \geq m\}$

\begin{solution}
    \digraph[scale=0.5]{fsm1d}{
        rankdir=LR;

        node [shape = point]; qi;
        node [shape = doublecircle]; q6;

        node [shape = circle];
        qi -> q0;
        q0 -> q1 [label="a"];
        q1 -> q2 [label="a"];
        q1 -> q1 [label="b"];
        q2 -> q2 [label="a,b"];
        q2 -> q3 [label="c"];
        q3 -> q4 [label="a"];
        q3 -> q5 [label="b"];
        q3 -> q3 [label="c"];
        q4 -> q6 [label="a"];
        q4 -> q4 [label="b"];
        q5 -> q4 [label="a"];
        q5 -> q5 [label="b"];
        q6 -> q6 [label="a"];
        q6 -> q4 [label="b"];
    }
\end{solution}

\part $T(M) = \{x \mid x \in (0, 1)^* \land x \text{ seja um número binário
    cujo decimal correspondente seja par e não divisível por } 3\}$

\begin{solution}
    \digraph[scale=0.5]{fsm1e}{
        rankdir=LR;

        node [shape = point]; qi;
        node [shape = doublecircle]; q1; q3; q4;

        node [shape = circle];
        qi -> q0;
        q0 -> q1 [label="0"];
        q0 -> q2 [label="1"];
        q1 -> q1 [label="0"];
        q1 -> q2 [label="1"];
        q2 -> q3 [label="0"];
        q2 -> q0 [label="1"];
        q3 -> q4 [label="0"];
        q3 -> q5 [label="1"];
        q4 -> q3 [label="0"];
        q4 -> q0 [label="1"];
        q5 -> q4 [label="0"];
        q5 -> q5 [label="1"];
    }
\end{solution}

\part $T(M) = \{(a, b)^*(c, d)^* \mid \#a’s + \#c’s \text{ é par} \land \#b’s +
    \#d’s \text{ é ímpar}\}$

\begin{solution}
    \digraph[scale=0.5]{fsm1f}{
        rankdir=LR;

        node [shape = point]; qi;
        node [shape = doublecircle]; q2; q5;

        node[shape = circle];
        qi -> q0;
        q0 -> q1 [label="a"];
        q0 -> q2 [label="b"];
        q1 -> q0 [label="a"];
        q1 -> q3 [label="b"];
        q2 -> q3 [label="a"];
        q2 -> q0 [label="b"];
        q3 -> q2 [label="a"];
        q3 -> q1 [label="b"];

        q0 -> q4 [label="c"];
        q0 -> q5 [label="d"];
        q1 -> q6 [label="c"];
        q1 -> q7 [label="d"];
        q2 -> q7 [label="c"];
        q2 -> q6 [label="d"];
        q3 -> q4 [label="c"];
        q3 -> q4 [label="d"];

        q4 -> q6 [label="c"];
        q4 -> q7 [label="d"];
        q5 -> q7 [label="c"];
        q5 -> q6 [label="d"];
        q6 -> q4 [label="c"];
        q6 -> q5 [label="d"];
        q7 -> q5 [label="c"];
        q7 -> q4 [label="d"];
    }
\end{solution}

\end{parts}

\question Construa a G.R. correspondente a 2 AFs obtidos no item anterior.

\begin{parts}

\part Para o autômato 1a:
\begin{solution}
    \begin{lstlisting}
        S -> aA | bB | cC | a | b | c
        A -> aS | bD | cE
        B -> cE
        C -> bD
        D -> cC | c
        E -> bB | b
    \end{lstlisting}
\end{solution}

\part Para o autômato 1b:
\begin{solution}
    \begin{lstlisting}[mathescape=true]
        S' -> 0S | 1A | 0 | $\varepsilon$
        S -> 0S | 1A | 0
        A -> 0B | 1C
        B -> 0D | 1E
        C -> 0A | 1S | 0
        D -> 0B | 1C
        E -> 0D | 1E
    \end{lstlisting}
\end{solution}

\end{parts}

\question Seja $T(M) = \{x \mid x \in (a, b)^* \land \text{o $n$-ésimo símbolo
da direita para a esquerda de x seja } b\}$. Pede-se: Qual o número de estados
(em função de $n$) de $M$ caso ele seja não determinístico? E se ele for
determinístico? Exemplifique.

\begin{solution}
    \begin{itemize}
        \item Um AF não-determinístico para este problema requer $n + 1$
            estados.

            Para $n = 2$:

            \digraph[scale=0.5]{fsm3a2}{
                rankdir=LR;

                node [shape = point]; qi;
                node [shape = doublecircle]; q2;

                node[shape = circle];
                qi -> q0;
                q0 -> q0 [label="a,b"];
                q0 -> q1 [label="b"];
                q1 -> q2 [label="a,b"];
            }

            Para $n = 3$.

            \digraph[scale=0.5]{fsm3a3}{
                rankdir=LR;

                node [shape = point]; qi;
                node [shape = doublecircle]; q3;

                node[shape = circle];
                qi -> q0;
                q0 -> q0 [label="a,b"];
                q0 -> q1 [label="b"];
                q1 -> q2 [label="a,b"];
                q2 -> q3 [label="a,b"];
            }

            E assim sucessivamente, adicionando um estado (como o novo $q2$) a
            cada incremento de $n$, com as transições apropriadas.

        \item Um AF determinístico para este problema requer $2^n$ estados. Os
            exemplos abaixo foram obtivos determinizando e minimizando os
            exemplos acima.

            Para $n = 2$:

            \digraph[scale=0.5]{fsm3b2}{
                rankdir=LR;

                node [shape = point]; qi;
                node [shape = doublecircle]; q2; q3;

                node[shape = circle];
                qi -> q0;
                q0 -> q0 [label="a"];
                q0 -> q1 [label="b"];
                q1 -> q2 [label="a"];
                q1 -> q3 [label="b"];
                q2 -> q0 [label="a"];
                q2 -> q1 [label="b"];
                q3 -> q2 [label="a"];
                q3 -> q3 [label="b"];
            }

            Para $n = 3$:

            \digraph[scale=0.5]{fsm3b3}{
                rankdir=LR;

                node [shape = point]; qi;
                node [shape = doublecircle]; q4; q5; q6; q7;

                node[shape = circle];
                qi -> q0;
                q0 -> q0 [label="a"];
                q0 -> q1 [label="b"];
                q1 -> q2 [label="a"];
                q1 -> q3 [label="b"];
                q2 -> q4 [label="a"];
                q2 -> q5 [label="b"];
                q3 -> q6 [label="a"];
                q3 -> q7 [label="b"];
                q4 -> q0 [label="a"];
                q4 -> q1 [label="b"];
                q5 -> q2 [label="a"];
                q5 -> q3 [label="b"];
                q6 -> q4 [label="a"];
                q6 -> q5 [label="b"];
                q7 -> q6 [label="a"];
                q7 -> q7 [label="b"];
            }

    \end{itemize}
\end{solution}

\question Construa um AFD Mínimo $M \mid T(M) = L(G)$ e determine $T(M)$, onde
$G$ é definida por:

\begin{parts}

\part
\begin{lstlisting}[mathescape=true]
    S -> aB | aD | bA | bC | a | b | $\varepsilon$
    A -> aB | bA | a
    B -> bB | aA | b
    C -> aD | bC | b
    D -> aC | bD | a
\end{lstlisting}

\begin{solution}
    Tabela de transições:

    \begin{tabular}{ r || c | c }
        $\delta$ & $a$ & $b$ \\ \hline\hline
        $\rightarrow\star$ q0 & q2, q4, q5 & q1, q3, q5 \\
        q1 & q2, q5 & q1 \\
        q2 & q1 & q2, q5 \\
        q3 & q4 & q3, q5 \\
        q4 & q3, q5 & q4 \\
        $\star$ q5 & --- & --- \\
    \end{tabular}\newline

    Transformando em AF determinístico:

    \begin{tabular}{ r || c | c }
        $\delta$ & $a$ & $b$ \\ \hline\hline
        $\rightarrow\star$ [q0] & [q2, q4, q5] & [q1, q3, q5] \\
        $\star$ [q2, q4, q5] & [q1, q3, q5] & [q2, q4, q5] \\
        $\star$ [q1, q3, q5] & [q2, q4, q5] & [q1, q3, q5] \\
    \end{tabular}\newline

    Classes de equivalência:

    \begin{itemize}
        \item $F = \{$[q0], [q2, q4, q5], [q1, q3, q5]$\}$
        \item $K - F = \varnothing$
    \end{itemize}

    Há somente 1 classe de equivalência, e não há como formar uma nova. O
    autômato finito determinístico mínimo resultante é:

    \digraph[scale=0.5]{fsm4a}{
        rankdir=LR;

        node [shape = point]; qi;
        node [shape = doublecircle]; q0;

        node[shape = circle];
        qi -> q0;
        q0 -> q0 [label="a,b"];
    }
\end{solution}

\part
\begin{lstlisting}[mathescape=true]
    S' -> aA | cC | bA | bC | $\varepsilon$
    S -> aA | cC | bA | bC
    A -> bS | cD | b | c
    C -> bS | aE | b | a
    D -> aA | bA | bC
    E -> cC | bC | bA
\end{lstlisting}

\begin{solution}
    Nota-se que as produções de $S$ e $S'$ são as mesmas, sendo que a segunda
    foi criada para evitar derivação com $\varepsilon$, e serão consideradas
    como um só terminal.

    Tabela de transições:

    \begin{tabular}{ r || c | c | c }
        $\delta$ & $a$ & $b$ & $c$ \\ \hline\hline
        $\rightarrow\star$ q0 & q1 & q1, q2 & q2 \\
        q1 & --- & q0, q5 & q3, q5 \\
        q2 & q4, q5 & q0, q5 & --- \\
        q3 & q1 & q1, q2 & --- \\
        q4 & --- & q1, q2 & q2 \\
        $\star$ q5 & --- & --- & --- \\
    \end{tabular}\newline

    Transformando em AF determinístico, renomeando estados:

    \begin{tabular}{ r || c | c | c || l }
        $\delta$ & $a$ & $b$ & $c$ & nome \\ \hline\hline
        $\rightarrow\star$ [q0] & [q1] & [q1, q2] & [q2] & q0 \\
        \null [q1] & --- & [q0, q5] & [q3, q5] & q1 \\
        \null [q1, q2] & [q4, q5] & [q0, q5] & [q3, q5] & q2 \\
        \null [q2] & [q4, q5] & [q0, q5] & --- & q3 \\
        $\star$ [q0, q5] & [q1] & [q1, q2] & [q2] & q4 \\
        $\star$ [q3, q5] & [q1] & [q1, q2] & --- & q5 \\
        $\star$ [q4, q5] & --- & [q1, q2] & [q2] & q6 \\
    \end{tabular}\newline

    Classes de equivalência:

    \begin{itemize}
        \item $F = \{$q0, q4, q5, q6$\}$
        \item $K - F = \{$q1, q2, q3$\}$
    \end{itemize}


    $\{$q0, q4, q5, q6$\}_0\{$q1, q2, q3$\}_1$ \\
    $\{$q0, q4, q5, q6$\}_0\{$q1, q2, q3$\}_1$
\end{solution}

\end{parts}

\end{questions}

\end{document}
